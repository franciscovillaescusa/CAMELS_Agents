\documentclass{article}
\begin{document}

CAMELS, an acronym for Cosmology and Astrophysics with MachinE Learning Simulations, is a comprehensive collection of over 15,000 cosmological simulations. This collection comprises 8,925 hydrodynamic simulations and 6,136 N-body simulations.

The hydrodynamic simulations within CAMELS are categorized into three distinct groups:

\begin{itemize}
\item \textbf{Generation}. Simulations in this group are classified based on the resolution, with the first, second, and third generations featuring $256^3$, $512^3$, and $1024^3$ dark matter particles and fluid elements, respectively, within a periodic cubic box of sizes $25$, $50$, and $100~h^{-1}{\rm Mpc}$.
\item \textbf{Suite}.  CAMELS encompasses nine distinct suites, each characterized by the specific code and model employed for the simulations. These suites are:

\begin{enumerate}
\item \textbf{IllustrisTNG}. These simulations utilize the AREPO code and the IllustrisTNG subgrid physics model. CAMELS contains 4,243 IllustrisTNG simulations.
\item \textbf{SIMBA}. These simulations use the GIZMO code and the SIMBA subgrid physics model. CAMELS contains 1,171 SIMBA simulations.
\item \textbf{Astrid}.  These simulations employ the MP-Gadget code and the ASTRID subgrid physics model. CAMELS contains 2,080 Astrid simulations.
\item \textbf{Magneticum}. These simulations utilize the OpenGadget code and the Magneticum subgrid physics model. CAMELS contains 77 Magneticum simulations.
\item \textbf{Swift-EAGLE}. These simulations use the Swift code and the EAGLE subgrid physics model. CAMELS contains 1,052 Swift-EAGLE simulations.
\item \textbf{Ramses}. These simulations are run with the Ramses code. CAMELS contains 296 Ramses simulations.
\item \textbf{CROCODILE}. These simulations use the Gadget4-Osaka code and the CROCODILE subgrid physics model. CAMELS contains 260 CROCODILE simulations.
\item \textbf{Obsidian}. These simulations employ the Gizmo code and the Obsidian subgrid physics model. CAMELS contains 27 Obsidian simulations.
\item \textbf{Enzo}. These simulations are run with the Enzo code. CAMELS contains 6 Enzo simulations.
\end{enumerate}

\item \textbf{Set}. The CAMELS simulations are further classified into six sets based on the arrangement of cosmological, astrophysical parameters, and initial random seeds.  For this paper, the relevant sets are:

\begin{itemize}

\item \textbf{SB}. This set comprises at least 128 simulations. Each simulation features a unique combination of cosmological and astrophysical parameters arranged in a Sobol sequence with $2^N$ elements, where $N$ is an integer. Additionally, each simulation has a distinct initial random seed. This set is denoted as SBX, where X represents the number of dimensions, e.g., SB28 for the IllustrisTNG suite. SB stands for Sobol sequence.

\item \textbf{CV}. This set consists of 27 simulations. All simulations share the same cosmological and astrophysical parameters (set to their fiducial values) and only differ in their initial conditions random seeds. This set is typically used to investigate the impact of cosmic variance. CV stands for Cosmic Variance.

\item \textbf{1P}. This set comprises four simulations per parameter plus one fiducial. Each simulation in this set varies one cosmological or astrophysical parameter at a time, while the random seed for initial conditions remains constant across all simulations. This set is used to study the influence of these parameters on a specific quantity. 1P stands for 1-parameter at a time.

\end{itemize}
\end{itemize}

The initial conditions for all simulations were generated at redshift $z=127$ using 2LPT. Each hydrodynamic simulation in CAMELS has a corresponding N-body counterpart.

Each simulation produces 91 snapshots, with several saved at various redshifts, including $z=0$. Halos and subhalos are identified using the Friends-of-Friends (FoF), Subfind, Rockstar, AHF, and CAESAR codes. Merger trees are extracted using Sublink and Consistent trees.



\end{document} 


